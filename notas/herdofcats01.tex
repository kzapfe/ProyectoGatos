\documentclass[a4paper,12pt]{article}

\usepackage[utf8]{inputenc}
\usepackage{amsmath,amssymb}
\usepackage{graphicx}
\usepackage{subfigure}
%\usepackage[spanish]{babel}
\usepackage{bm}

\usepackage[cm]{fullpage}
\usepackage[light]{antpolt}
\usepackage[T1]{fontenc}


\bibliographystyle{alpha}

\newcommand{\ihb}{\frac{i}{\hbar}}
\newcommand{\xfase}{\mathbf{x}}
\newcommand{\qfase}{\mathbf{q}}
\newcommand{\pfase}{\mathbf{p}}
\newcommand{\xifase}{ {\boldsymbol{\xi}} }
\newcommand{\mufase}{ {\boldsymbol{\mu}} }
\newcommand{\Ifase}{\mathbf{I}}
\newcommand{\Pfase}{\mathbf{P}}
\newcommand{\Scat}{\mathbf{S}}
\newcommand{\Jsimp}{\mathbf{J}}
\newcommand{\Dom}{\mathbb{D}}
\newcommand{\Var}{\mathbb{M}}
\newcommand{\bra}[1]{\langle #1|}
\newcommand{\ket}[1]{|#1\rangle}
\newcommand{\braket}[2]{\langle #1|#2\rangle}
\newcommand{\Prom}[2]{\langle #1\rangle_{#2}}


\DeclareMathOperator*{\cod}{cod}
\DeclareMathOperator*{\traza}{traza}


\title{A Herd of Cats State and its Chord Function Representation}
\author{\\CBPF}


\begin{document}

\maketitle

\begin{abstract}

\end{abstract}


\section{A new proposal for  Weak Ergodic Wigner Functions}

Recently we proposed a semiclassical approximation to 
a stationary state of a chaotic (ergodic, hyperbolic)
Hamiltonian system \cite{Deltas}.
Our approach tried to enrich the structure of the Wigner
conjecture by randomizing its ``coarseness''.  
The usual approach was based on the Voros and Berry
Conjecture \cite{Voros76, BerryRIR}, which states that
the limit for the Wigner or center function of a classically
chaotic (in the strong sense) Hamiltonian system would be
a Dirac Delta in the Energy shell, that is:
\begin{equation}
W(\xfase)=\frac{1}{N_E} \delta(H(\xfase)-E), 
\end{equation}
where $1/N_E$ is the normalization factor for the 
shell with energy  $E$. This approach tends to bypass
many important objections, such as Hudson's Theorem
\cite{Hudson}, which states that the only non negative
Wigner function is the one of the Gaussian coherent state.
Our most serious concern was not that, but the lack
of fine structure in such approach, a feature which
encompasses rich information about quantum effects
such as self interference and correlations. 

Thinking first on remedying such a undesirable feature,
we proposed an ultra-coarse grain approach to the Wigner
Function, with the condition that a weak and pragmatic
limit be respected:

\begin{equation}
\chi_E(\xifase)\rightarrow N_E 
\int d \xfase \exp\left(\ihb \xifase \wedge \xfase) \right)
\delta(H(\xfase)-E)
\end{equation}

The limit is for small chords $\xifase$ and small Planck's
Constant $\hbar$.

The first try was to simply simulated a coarse grained
distribution on the energy surface by randomly putting
punctual Dirac Deltas over it. That was meant to represent
the very scared appearance of a chaotic Wigner function.
Its symplectic Fourier Transform would be a random superposition
of sinusoidal waves, the exact distribution of them dependent
on the geometric qualities of the energy shell, as 
it's Taylor expansion reveals:

\begin{equation}
\chi(\xifase)\approx 
1-\ihb \xifase \wedge \Prom{\hat{\xfase}}{E}
-\frac{1}{\hbar^2}
\Prom{(\xifase\wedge \hat{\xfase})^2}{E}+\cdots
\end{equation}
The classical cumulants depend exclusively on the shape of the energy
shell, while the quantum mechanical expectation values should only
coincide with the former in the small $\hbar$ limit.
This has not worked out as it should and we still do not know why.
%This approach turn to be extremely sensitive to
%errors or inhomogeneities in the 
%distribution of the ``centers'', and we still have 
%a real problem with the quantum correspondence.

\section{The Cat Herd State}

\subsection{Chord Function for a Cat}

In order to honor Hudsons Theorem, we will make a computationaly
heavier approach: to superpose a lot of coherent states
in a pure state which I shall call the ``Herd of Cats'' or
``Cat Herd State''. The idea is that the superposition of two
Gaussian Coherent States produces in the 
phase space a Wigner Function which shows interference
patterns \cite{Vallejos10}. We could expect that a 
fair amount of coherent states with peaks randomly distributed over
the energy shell would mimic a scared Wigner Function,
with all its structural richness and negative values,
but still having a homogeneity on the large scale.

In order to do this, first we well agree in the basic
characteristics of this distribution. To set things clear,
we will specify what we understand by a Gaussian Coherent
state and a Herd of Cats state.   

Following the notation of \cite{Vallejos10}, we will
call a Gaussian Coherent State a quantum state which
corresponds to the displacement and squeezing of 
the ground state of an harmonic oscillator. The
Squeezing would be indexed by the classical symplectic
linear matrix $S$, and the translation by the 
chord $\xifase$.  The Gaussian Coherent State would be:
\begin{equation}
\ket{S,\xifase}=\hat{T_\xifase}\hat{M_S}\ket{0},
\end{equation}
where the operators are the abstract quantum
operators corresponding to aforementioned classical transformations.

The Wigner Function for such a state is a Gaussian
with center $\xifase$ and spread given by $S$, the covariance 
Matrix being $SS^t$. To agree in a convenient notation
I shall use the following notation for the 
Pure Gaussian Coherent State Wigner Function:
\begin{equation}
W_{S,\xifase}(\xfase)=\frac{\sqrt{\det (SS^\dagger)^{-1}}}{(\pi\hbar)^d}
\exp(-(\xfase-\xifase)(SS^\dagger)(\xfase-\xifase)/\hbar)
\end{equation}
Here $d$ is the number of degrees of freedom. It shall be noticed
that $S$ and $SS^\dagger$ are symplectic, so the expression simplifies
as $\det SS^\dagger =1$. 

A Gaussian Cat State is simply the state corresponding
to the superposition of two Gaussian Coherent States.
Its Wigner Function has a term for each of the 
coherent states identical to the one obtained by such a state
alone, and one  interference term. Let $\ket{ A, \xifase}$
and $\ket{ B, \mufase}$ be the two coherent components in question,
and they are equally weighted and properly normalized.
Then its Wigner Function would be:
\begin{equation}
W_{cat}(\xfase)=W_{A,\xifase}(\xfase)+W_{B,\mufase}(\xfase)+
WI_{A,B,\xifase,\mufase}(\xfase).
\end{equation}
The interference term is a bit cumbersome to write down (from here on
it shall be denoted by a capital $I$), but it has a 
very clear geometrical meaning. It is a Gaussian enveloped
oscillating function, which comes from the sum of the terms 
$\ket{A,\xifase} \bra{B, \mufase}$ and
$\ket{B,\mufase} \bra{A, \xifase}$ of the density matrix. 
If we make the following abbreviations the structure of this
interference term becomes evident:

\begin{align}
\zeta & = \mufase-\xifase \\
\eta & = (\mufase+\xifase)/2 \\
K & =\frac{2^d i^\nu}{\sqrt{\det(A+B)+i(B-A)J}} \\
G & = (BB^\dagger+AA^\dagger)^{-1}(2-i(BB^\dagger -AA^\dagger)J)
\end{align}
$J$ is the symplectic matrix.
The interference term would be:

\begin{equation}
WI_{A,B,\mufase,\xifase}(\xfase)=
2 |K| \Re \left( \exp (\ihb \xfase \wedge \zeta +i\phi) \right) \\ 
\times \exp\left(-(\xfase-\eta)G(\xfase-\eta)/\hbar \right)
\frac{\sqrt{\det G}}{(\pi\hbar)^d}
\end{equation}

The Gaussian envelope term would correspond to a Wigner Coherent State
Function for a Correlation Matrix such that $SS^\dagger=G$.

We are interested
on obtaining the chord function of such a term, so we investigate
the symplectic Fourier Transform of it.
As this term is the sum of the contributions from the cross terms
of the density matrix, for simplicity we transform each one separately.
We choose $\lambda$ as the chord for the Fourier Transform.
Leaving multiplicative constants aside, we have:
\begin{multline}
\int d \xfase \exp \left(\ihb (\lambda-\zeta)\wedge \xfase) \right)
\exp \left( -(\xfase-\eta)G (\xfase-\eta)/\hbar \right) =\\
\exp \left(\ihb (\lambda-\zeta)\wedge \eta) \right)
\exp \left( -(\lambda-\eta)G^{-1}(\lambda-\eta)/\hbar \right),
\end{multline}
and
\begin{multline}
\int d \xfase \exp \left(\ihb (\lambda+\zeta)\wedge \xfase) \right)
\exp \left( -(\xfase-\eta)G (\xfase-\eta)/\hbar \right) =\\
\exp \left(\ihb (\lambda+\zeta)\wedge \eta) \right)
\exp \left( -(\lambda+\eta)G^{-1}(\lambda+\eta)/\hbar \right).
\end{multline}
For the case $G=1$ these coincide with Zambrano's results 
\cite{tesiseduardo}.

\subsection{Chord Function for a Cat Herd: simple tests}

Now the procedure is to populate the energy shell with cat states. For that
we will procede in the following way. We will choose randomly a relative
small set of Gaussian Coherent States which would have center very
close to the energy shell of the problem. ``Very close'' is
less than  one unit of the minimal quantal unit of energy of the problem.
If we choose all contributors to the state to have the same wheight,
the density matrix will look like:
\begin{equation}
\hat{\rho}=\frac{1}{N^2}\sum_{k,j=1}^{N}
\exp(i(\phi_k-\phi_j))\ket{A_k,\eta_k}\bra{A_j, \eta_j}
\end{equation}
with certain relative phases $\phi_n$.

The Wigner Function for the Herd of Cats would be
\begin{equation}
W_{herd}(\xfase)=\frac{1}{N}\sum_k W_k(\xfase)
+\frac{2}{N(N-1)}\sum_{k\neq j}WI_{k,j}(\xfase),
\end{equation}
that is, the individual Wigner Functions for each coherent
component indexed by $k$ and the interference terms for each of the
posible pairs indexed by $(k,j)$. The Chord Function will have
the same structure, by linearity of the SFT. 
\begin{equation}
\chi_{herd}(\xifase)=\frac{1}{N}\sum_k \chi_k(\xifase)
+\frac{2}{N(N-1)}\sum_{k\neq j}\chi I_{k,j}(\xfase).
\end{equation}

We shall produce a  test of our Herd of Cats over a very simple
Energy Shell, namely, a circle, the Energy Shell of a 1D Harmonic Oscillator,
as we did two years ago with the points, and compare it with the corresponding
Bessel Function \cite{BerryRIR}. This result shall also give us an idea of the 
time needed for obtaining the desired chord functions for more complex
Hamiltonians. We have choosen relative large energy shells (E=) for the test,
and various Herd \emph{sizes} ($N=$). The results are shown in the
following figures (\ref{}, ), in them  we can observe certain encouraging 
qualities. A most delightfull feature on the graphics below is that we can
see something akin to an  Arago Spot \cite{Arago} 
close to the midle of the circle, the only point
on the graphics where our convention allows for mostly constructive 
interference. With few Coherent States the spot is brighter than the
surrounding ring, but as cats acumulate, we begin to have negative values
for the centre function very close to it, thus dulling part of its
preponderance. 


\section{A Herd of Cats over Nelson}

To compare this approach with the pointillist one, I have choosen the centers
of the Gaussian States to coincide with the Dirac Deltas. 
In principle this should give me more or less the same limiting behavior for 
very narrow coherent states.  This poses a question. 
How shall we determine the narrowness of the coherent components of the cats-states? 
A coherent state has three parameters which determine its width:
the mass of the particle, $m$, the oscilating frecuency $\omega$, and $\hbar$. 
The last one is fixed for all our disertations. 
The mass can be chossen to be $1$, as it is in our Nelson Hamiltonian Function.





\bibliography{herdofcats}

\end{document}



